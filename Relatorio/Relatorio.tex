\documentclass[a4paper,10pt]{article}
\usepackage[brazil]{babel}
\usepackage[utf8]{inputenc}
\usepackage[T1]{fontenc}
\usepackage{amsmath}
\usepackage{hyperref}
\usepackage{url}
\usepackage{graphicx}
\usepackage{subfigure}

\title{Sistema de Cinema}
\author{
Felipe Luís Pinheiro - 18/0052667 \and 
João Pedro C.N. Mota - 17/0106144 \and 
Pedro Catelli - 17/0112624 \and
Pedro Oliveira - 17/0163768}

\begin{document}

\maketitle

\begin{abstract}
Neste relatório desenvolvemos os requisitos básicos de um sistema de banco de dados para um modelo de vendas de ingresso de um cinema. 
\end{abstract}

\section{Projeto Cinema}

Requisitos gerais:

\begin{itemize}
	\item Um cinema pode ter muitas salas, sendo necessário, por tanto, registrar informações a respeito de cada uma, como sua capacidade, ou seja, o numero de assentos disponíveis.
	\item O cinema apresenta muitos filmes. Um filme tem informações, titulo e duração. Assim, sempre que um filme for ser apresentado, deve-se registrá-lo também.
	\item Um mesmo filme pode ser apresentado em diferentes salas e em horários diferentes. Cada apresentação em uma determinada sala e horário é chamada sessão. Um filme sendo apresentado em uma sessão tem um conjunto máximo de ingressos, determinado pela capacidade da sala.
	\item Os clientes do cinema podem comprar ou não ingressos para assistir a uma sessão. O
funcionário deve intermediar a compra do ingresso. Um ingresso deve conter informação
como o tipo de ingresso (Meio ingresso ou ingresso inteiro). Além disso, um cliente só pode
comprar ingressos para sessões ainda não encerradas.
\end{itemize}


%\pagebreak
%\bibliographystyle{plain}
%\bibliography{MyLib} 
\end{document} 